\documentclass[a4paper,english,12pt]{article}
\usepackage[utf8]{inputenc}
\usepackage[english]{babel}
\usepackage{times}
\usepackage[T1]{fontenc}

\usepackage{parskip}
\usepackage{amsmath}
\usepackage{amssymb}
\usepackage{varioref}
\usepackage{subcaption}
\usepackage{units}
\usepackage{todo}
\usepackage[super]{nth}
\usepackage{fancyvrb}
\usepackage{verbatim}
\usepackage{xfrac}
\usepackage{float}
\usepackage{url}
\usepackage{booktabs}

% Double brackets
\usepackage{stmaryrd}
\usepackage[hidelinks]{hyperref}

% Footnote symbols
\usepackage[symbol*]{footmisc}

% \mathscr
\usepackage{mathrsfs}

%\usepackage{oving}

\DeclareUnicodeCharacter{00A0}{ }
\newcommand{\strek}{\textthreequartersemdash}

% Verbatim listings
\usepackage{listings}
\lstset{
  basicstyle=\small\ttfamily,
  columns=fullflexible,
  keepspaces=true,
}


\title{DAT-510 Assignment 2}
\author{Christian Stigen}
\date{UiS, October \nth{9}, 2017}

\begin{document}
\maketitle

\begin{abstract}
\end{abstract}

\section{Introduction}
The objective in this project is to create a complete program for performing
key exchange and send encrypted data between two actors.

\section{Design and implementation}

\subsection{Choice of programming language}
Writing a secure, production-ready cryptography system is a serious
undertaking. While security is paramount, such a system should operate quickly
as well. This being a student project, I have chosen flexibility above
speed. Aspects concerning the security of the implementation have been noted
throughout this report.

I have chosen to use Python as the sole programming language, mostly because it
comes with big integer support built-in. A faster language, such as C or C++,
would require to use the GMP library \cite{wiki:gmp} for implementing bignums.
This requires installing additional packages, and I have therefore chosen not
to use those languages in this project.

\subsection{Choice of global parameters}
While the implementation contains code for choosing $p$ and $q$ for the
Diffie-Hellman key exchange, I have chosen numbers, with a generator $g$, from
\cite{rfc3526}. The arguments were originally made for use with the IKE key
exchange protocol (using Diffie-Hellman), and have been found ot have good
number theoretic security properties.

\subsection{Choice of CSPRNG}
I have chosen to use the Blum Blum Shub \cite{bbs} pseudo-random number
generator. This section will give details on the implementation.

The first step is to find two large prime numbers $p$ and $q$. I have
implemented the Miller-Rabin \cite{miller.rabin} primality test to generate
them.
%
If the bit length is 512, I select a random integer between $2^{512-1}$ and
$2^{512}-1$, then perform the Miller-Rabin test with an appropriate accuracy
setting. There are two things to consider here: The choice of pseud-random
number generator (PRNG) for choosing a candidate, and the accuracy setting of
the Miller-Rabin test.

\paragraph{The PRNG} for choosing a candidate prime is done with the built-in
\texttt{random.randint} function in Python. This is not a cryptographically
secure pseud-random number generator (CSPRNG). One could investigate whether a
non-CSPRNG would weaken the BBS algorithm. But the number of primes between
$2^{512-1}$ and $2^{512}-1$ is \textit{huge} \cite{wiki:prime.counting}, so I
have chosen not to investigate this issue any further.

\paragraph{The accuracy setting} is discussed in \cite{damgaard1993average}, which
provides a method to select the accuracy based on a desired error bound. I have
not implemented an automatic selection of the accuracy parameter in the code,
however. In the paper, they state that a $k=600$-bit number with $t=1$
iteration, the upper bound for the probability that the Miller-Rabin test
actually produces a composite number is $p_{600, 1} \leqslant 2^{-75}$.

If the test is negative, I try another number. If it is true, I accept
it. However, the Miller-Rabin primality test is \textit{probabilistic}. While
it will always be correct if it says a number is not prime, it may give false
positives, saying a number is \textit{probably} prime, but cannot be certain
about it. At this point in the code, I do not perform any further checks on the
number. This is a weakness of the algorithm that should be addressed in a
production system.

The primes $p$ and $q$ must be congruent to $4 (mod 3)$, meaning that $p mod 4
== 3$ and likewise for $q$.

\section{Test results}

\section{Discussion}

\section{Conclusion}

\bibliography{assignment-2}
\bibliographystyle{ieeetr}

\end{document}
